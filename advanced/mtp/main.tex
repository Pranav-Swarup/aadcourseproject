\documentclass[11pt, a4paper]{article}

% Packages for formatting and mathematics
\usepackage[utf8]{inputenc}
\usepackage{geometry}
\usepackage{amsmath, amssymb, amsthm}
\usepackage{algorithm}
\usepackage{algpseudocode}
\usepackage{graphicx}
\usepackage{hyperref}

% Geometry settings
\geometry{left=25mm, right=25mm, top=25mm, bottom=25mm}

% Theorem environments
\newtheorem{theorem}{Theorem}
\newtheorem{lemma}{Lemma}
\newtheorem{definition}{Definition}
\newtheorem{corollary}{Corollary}

% Title Information
\title{\textbf{The Martello-Toth Procedure (MTP) for the One-Dimensional Bin Packing Problem}}
\author{A Detailed Algorithmic Analysis}
\date{\today}

\begin{document}

\maketitle

\begin{abstract}
The One-Dimensional Bin Packing Problem (1DBPP) is an NP-hard combinatorial optimization problem. While heuristic approaches provide approximate solutions, exact algorithms are required to determine the minimal number of bins $m$. This document details the Martello-Toth Procedure (MTP), an exact Branch-and-Bound algorithm. We rigorously define the bounding procedures ($L_1$ and $L_2$), the dominance criteria used for problem reduction, and provide the mathematical proof of correctness for the $L_2$ bound, which serves as the primary pruning mechanism for the search tree.
\end{abstract}

\section{Problem Definition}
Given a set of $n$ items with integer sizes $I = \{s_1, s_2, \dots, s_n\}$ and a fixed bin capacity $C$, the objective is to partition $I$ into the minimum number of subsets (bins) $B_1, \dots, B_m$ such that the sum of sizes in each bin does not exceed $C$.

We assume without loss of generality that items are sorted in non-increasing order of size:
\begin{equation}
s_1 \ge s_2 \ge \dots \ge s_n
\end{equation}

\section{Mathematical Lower Bounds}
The efficiency of the MTP depends on the tightness of its lower bounds. A lower bound $L(I)$ allows the algorithm to prune branches of the search tree where the current partial solution plus the lower bound of the remaining items exceeds the best known solution.

\subsection{The $L_1$ Bound (Volume Bound)}
The simplest lower bound is derived from the continuous relaxation of the problem (assuming items can be split like liquid).
\begin{equation}
L_1 = \left\lceil \frac{\sum_{i=1}^n s_i}{C} \right\rceil
\end{equation}
It is proven that the worst-case performance ratio of $L_1$ is $1/2$. While fast ($O(n)$), it is often loose for instances where items are large relative to $C$.

\subsection{The $L_2$ Bound (Martello-Toth Bound)}
The core mathematical contribution of the MTP is the $L_2$ bound, which improves upon $L_1$ by analyzing "wasted" space that is mathematically unavoidable.

\subsubsection{Definitions}
For any integer parameter $K$ such that $0 \le K \le C/2$, we classify the items into three sets:
\begin{align*}
N_1(K) &= \{ i \in I : s_i > C - K \} & \text{(Large items)} \\
N_2(K) &= \{ i \in I : C - K \ge s_i > C/2 \} & \text{(Medium items)} \\
N_3(K) &= \{ i \in I : C/2 \ge s_i \ge K \} & \text{(Small items)}
\end{align*}

\subsubsection{The $L(K)$ Function}
Based on these sets, we define a function $L(K)$:
\begin{equation}
L(K) = |N_1| + |N_2| + \max\left(0, \left\lceil \frac{\sum_{i \in N_3} s_i - R_{N_2}}{C} \right\rceil \right)
\end{equation}
where $R_{N_2}$ is the total residual capacity left in the bins containing items from $N_2$:
\[
R_{N_2} = |N_2|C - \sum_{i \in N_2} s_i
\]

\begin{theorem}[Correctness of $L(K)$]
For any $K \in [0, C/2]$, $L(K)$ is a valid lower bound on the optimal number of bins $m$.
\end{theorem}

\begin{proof}
Consider the packing requirements of the sets $N_1$, $N_2$, and $N_3$:
\begin{enumerate}
    \item \textbf{Separation of Large Items:} Every item in $N_1$ has size $s_i > C-K$. Every item in $N_2$ has size $s_i > C/2$. Since $K \le C/2$, all items in $N_1 \cup N_2$ have size strictly greater than $C/2$. Therefore, no two items from $N_1 \cup N_2$ can fit into the same bin. This implies that at least $|N_1| + |N_2|$ bins are required just to hold these items.
    
    \item \textbf{Incompatibility of $N_1$ and $N_3$:} An item from $N_3$ has size $s_i \ge K$. An item from $N_1$ has size $s_j > C-K$. The sum $s_i + s_j > C$. Thus, no item from $N_3$ can be placed in a bin containing an item from $N_1$.
    
    \item \textbf{Filling the Gaps:} The items in $N_3$ must be packed either into the remaining space of bins containing $N_2$ items, or into completely new bins.
    The total available space in the bins utilized by $N_2$ items is exactly $R_{N_2} = |N_2|C - \sum_{i \in N_2} s_i$.
    
    \item \textbf{Calculation:} The total size of items in $N_3$ is $\sum_{i \in N_3} s_i$. The portion of this total size that \textit{cannot} fit into the $N_2$ bins is:
    \[
    \text{Excess} = \max\left(0, \sum_{i \in N_3} s_i - R_{N_2}\right)
    \]
    This excess volume must go into new bins. The minimum number of additional bins required for this excess is $\lceil \text{Excess} / C \rceil$.
    
    \item \textbf{Conclusion:} The total bins required is the sum of bins for $N_1 \cup N_2$ plus the additional bins for the overflow of $N_3$.
    \[
    m \ge (|N_1| + |N_2|) + \left\lceil \frac{\max(0, \sum_{N_3} s_i - R_{N_2})}{C} \right\rceil
    \]
\end{enumerate}
This concludes the proof.
\end{proof}

\subsubsection{The Optimized $L_2$ Bound}
The bound $L_2$ is defined as the maximum value of $L(K)$ over all feasible $K$:
\begin{equation}
L_2 = \max_{0 \le K \le C/2} L(K)
\end{equation}
To compute this efficiently, it is sufficient to check $K$ only for distinct values of $s_i \le C/2$. If items are sorted, this can be computed in $O(n)$ time.

\section{Reduction Procedures}
Before and during the branching process, MTP applies reduction procedures to fix bins permanently, reducing the problem size. This relies on the concept of \textbf{Dominance}.

\begin{definition}[Feasible Set Dominance]
A feasible set $F_1$ dominates a feasible set $F_2$ if the optimal solution obtained by setting a bin $B = F_1$ is not worse than the optimal solution obtained by setting $B = F_2$.
\end{definition}

\subsection{Dominance Criterion}
A sufficient condition for dominance is:
If $F_1$ and $F_2$ are distinct feasible sets, and there exists a partition $P = \{P_1, \dots, P_\ell\}$ of $F_2$ and a subset $\{i_1, \dots, i_\ell\} \subseteq F_1$ such that:
\begin{equation}
s_{i_h} \ge \sum_{k \in P_h} s_k \quad \text{for } h=1, \dots, \ell
\end{equation}
then $F_1$ dominates $F_2$.

\subsection{Reduction Algorithm}
The MTP reduction algorithm (Procedure REDUCTION) iteratively looks for dominating sets to fix into bins.
\begin{algorithmic}[1]
\State Initialize fixed bins count $j := 0$, Unpacked set $I$
\Repeat
    \State Let $i$ be the largest remaining item.
    \State Find feasible set $F$ containing $i$ that dominates all other sets containing $i$.
    \State \textbf{Check 1 (Single Item):} If $i$ cannot fit with any other item, $F=\{i\}$.
    \State \textbf{Check 2 (Pairs):} If $i$ fits with $i'$, and $\{i, i'\}$ fills the bin so well (e.g., $s_i + s_{i'} = C$) that no triplet could be better, $F=\{i, i'\}$.
    \If{$F \neq \emptyset$}
        \State Fix Bin $B_{j+1} := F$.
        \State Remove items in $F$ from $I$.
    \EndIf
\Until{No further reductions possible}
\end{algorithmic}
This procedure essentially greedily matches items if they form a "perfect" or "dominant" bin, reducing the complexity for the subsequent Branch-and-Bound phase.

\section{The Exact Branch-and-Bound Algorithm}
The complete Martello-Toth Procedure combines the bounds and reductions into a Depth-First Search (DFS).

\subsection{Algorithm Steps}
\begin{enumerate}
    \item \textbf{Initialization:}
    \begin{itemize}
        \item Sort items $s_1 \ge s_2 \dots \ge s_n$.
        \item Calculate global lower bound $LB = L_2$.
        \item Calculate heuristic upper bound $UB$ (using First-Fit Decreasing or Best-Fit Decreasing).
        \item If $LB == UB$, the heuristic solution is optimal. STOP.
    \end{itemize}

    \item \textbf{Reduction:}
    \begin{itemize}
        \item Apply Procedure REDUCTION to fix easy bins. Update problem size and $LB$.
    \end{itemize}

    \item \textbf{Branching (Backtracking):}
    \begin{itemize}
        \item MTP builds a solution bin by bin.
        \item For the current bin, it attempts to place the largest available item.
        \item It then recursively attempts to fill the remaining capacity of the current bin with the next largest fitting items.
        \item Once a bin is closed, it moves to the next bin.
    \end{itemize}

    \item \textbf{Pruning (The Critical Step):}
    At any node in the search tree:
    \begin{itemize}
        \item Let $m_{current}$ be the number of bins already fixed/filled.
        \item Let $I_{rem}$ be the set of remaining unpacked items.
        \item Calculate the lower bound for the remainder: $L_2(I_{rem})$.
        \item \textbf{Condition:} If $m_{current} + L_2(I_{rem}) \ge UB$, then this branch cannot lead to a better solution than what we already found. \textbf{PRUNE (Backtrack)}.
    \end{itemize}

    \item \textbf{Updating Best Solution:}
    If a valid packing is found with $z$ bins and $z < UB$:
    \begin{itemize}
        \item Update $UB = z$.
        \item If $UB == LB$, STOP (Optimal found).
    \end{itemize}
\end{enumerate}

\section{Complexity and Optimality}
The MTP is an exact algorithm. While the worst-case time complexity is exponential (due to the NP-hardness of BPP), the effective use of the $L_2$ bound allows it to solve many instances efficiently. The $L_2$ bound has an asymptotic worst-case performance ratio of $2/3$, meaning it is tighter than simple volume bounds.

\end{document}
